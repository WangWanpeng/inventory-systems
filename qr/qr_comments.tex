\documentclass{article}

\usepackage{mathtools,amssymb}
\usepackage{fouriernc}
\newcommand{\E}{\mathbb{E}}
\newcommand{\1}[1]{\mathbb{1}_{#1}}
\renewcommand{\P}{\mathbb{P}}

\title{Cost Equations for  Q,r model}
\author{Nicky D. Van Foreest}
\begin{document}
\maketitle

Here we summarize some notation and functions used by Hadley and
Whitin (Section 4.7 for the analysis and Section 4.10 for a numerical
example) and Federgruen and Zheng (OR, 1991) in their descriptions of
the $(Q,r)$-inventory control rule. We refer to these places for
further background.

The long-run average cost is given by
\begin{equation*}
  C(r, Q) = K \frac{\lambda}{Q} + \frac1Q \sum_{y=r+1}^{r+Q} G(y),
\end{equation*}
where
\begin{equation*}
  G(y) = \E f(y-X)
\end{equation*}
and $f(y)$ is a cost function when the inventory level is $y$ and $X$
is the leadtime demand. To capture the average inventory cost, take
\begin{equation*}
   f(y) = h [y]^+,
\end{equation*}
for the backorder cost per unit per unit time, let
\begin{equation*}
   f(y) = b [-y]^+.
\end{equation*}
The loss fraction is given by
\begin{equation*}
   f(y) = \1{y\leq 0}.
\end{equation*}
To see this last equation, note that $\P(X\geq y)$ is the
fraction of demand lost, by PASTA. Clearly, for this $f$,
\begin{equation*}
\P(X\geq y) = \P(y-X \leq 0) = \E \1{y-X\leq 0} = \E f(y-X) = G(y).
\end{equation*}
Thus,  the cost per backorder becomes
\begin{equation*}
   f(y) = \lambda \pi 1\{y\leq 0\},
\end{equation*}
where $\pi$ is the cost per backordered demand and $\lambda$ the
arrival rate. The total cost follows by summing all these costs to
\begin{equation*}
   f(y) = h [y]^+ + b[-y]^+ + \lambda \pi 1\{y\leq 0\}.
\end{equation*}


To compute the cost associated with each of the separate components,
set $K=0$ in $C(r,Q)$ above, since, for instance,the average holding
costs should not include ordering costs.



We next relate the notation of Federgruen and Zheng to the
notation of Hadley and Whitin.

\begin{align*}
     \lambda &= \text{Arrival rate of demand}, \\
     L &= \tau = \text{The replenishment lead time}, \\
     X &= \text{The stochastic demand during the replenishment lead time}, \\
     F &= \text{The distribution of } X,\\
     Q &= \text{The ordering quantity}, \\
     r &= \text{Reorder level}, \\
     A & = K = \text{Ordering cost},\\
     \pi & = \text{Stockout cost per unit},\\
     \hat \pi & = \text{Stockout cost per unit per unit time},\\
     C &= \text{Item Cost}. \\
     I &= \text{Inventory carring charge }. \\
     h &= C*I = \text{Inventory cost per unit per unit time}. \\
\end{align*}


Finally we have some remarks on two formulae of Hadley and Whitin. (In
our numerical work we got other values than theirs\ldots)
\begin{enumerate}
\item The formula Eq. 4.93 in Hadley and Whitin for the average
  holding cost uses $Q^*$ but it should be $(Q^*+1)/2$. 
\item The value of $\pi E(Q,r) = 12.341$ as reported by Hadley and
  Whitin in Eq. 4.94, is not the same as our value. However, our value
  is the same as the one computed by the methods presented in Factory
  Physics.
\end{enumerate}

\end{document}



%%% Local Variables:
%%% mode: latex
%%% TeX-master: t
%%% End:
